\chapter{Introdução}

\section{Área de Estudo}
\label{sec:studyArea}

% §.1 localizacao da PCSE e características geográficas, principais cidades, topografia, profundidade de quebra do talue, profundidade media
\hspace{5mm}A Plataforma Continental Sudeste (PCSE), localizada entre Cabo Frio (23ºS), no RJ, e o Cabo de Santa Marta (28º40'S), em SC, é parte integrante do Oceano Atlântico Sudoeste,
estando em constante troca de massa e energia com o oceano profundo adjacente. Sua largura máxima ocorre ao largo de Santos (SP), com cerca de 230km e mais estreita nas extremidades,
chegando a 50km em Cabo Frio (Figura~\ref{fig:areaestudo}). A quebra da plataforma ocorre a cerca de 150km da costa e com profundidade variando entre 120 e 180m \citep{zembruscki1979geomorfologia}. A orientação da linha de costa é NE-SW, sofrendo uma mudança brusca de orientação na região de Cabo Frio (RJ), onde a linha de costa passa a ter uma orientação E-W, sendo que as isóbatas acompanham essa orientação, descrevendo uma topografia suave e com um volume total estimado de 10.000$km^3$, considerando uma profundidade média de 70m \citep{castro1996correntes}.

\hspace{5mm} A PCSE apresenta características típicas de uma plataforma do tipo A ("\textit{wide shelf with shelf-edge western boundary current}"), com a Corrente do Brasil influenciando a parte mais externa da plataforma (sobre a quebra do talude), mas não tendo influência nas regiões mais costeiras \citep{castro2014summer,loder1998western}.

% §.2 importância economica para o Brasil
\hspace{5mm} Devido a sua extensão, a PCSE é uma zona de enorme valor econômico para o Brasil, uma vez que diversos empreedimentos costeiros estão instalados nesta região, como o Porto de Santos, estações de extração de petróleo e gás, indústrias e demais exemplos atrelados ao desenvolvimento do país. Sendo assim, torna-se de suma importância compreendermos os mecanismos que controlam a circulação da plataforma e suas interações, a fim de mitigar quaisquer intervenções, atrópicas ou não, na região.

%\section{Massas de Água da PCSE}

\hspace{5mm} Ao largo da PCSE, a Corrente do Brasil transporta, basicamente, duas massas de água:

\begin{itemize}
    \item Água Tropical (AT), transportada em profundidades de até 200m, possui temperatura superior a 20ºC e salinidade superior a 36 \citep{miranda1982analise} e
\item Água Central do Atlântico Sul (ACAS), transportada abaixo da picnoclina (entre 200 e 600m), com temperatura inferior a 20ºC e salinidade abaixo de 36.40 \citep{miranda1982analise}.
\end{itemize}

\hspace{5mm} Devido a diversos mecanismos, essas massas podem deslocar-se em direção a costa e, devido ao aporte de água doce do continente, a mistura entre as massas resulta na Água Costeira (AC), caracterizada por baixa salinidade. Sendo assim, são três massas d'água que preenchem a PCSE (Figura~\ref{fig:massas_dagua}), onde a AC ocupa a região mais costeira, a AT ocupa níveis mais superficiais da coluna d'água e, por fim, a ACAS ocupa os níveis mais inferiores \citep{emilsson1961shelf,de1979aplicaccao,miranda1982analise,castro1998physical}.

\begin{figure}[!h]
    \centering
    \includegraphics[width=0.5\linewidth]{figuras/massasPCSE.png}
\caption[Massas d'água que preenchem a PCSE]{Esquema gráfico das massas d'água que preenchem a PCSE e suas principais características. A região rachurada representa águas da plataforma continental, que seria a mistura das três massas, mas sem uma definição precisa. Extraído de \cite{castro2015191}.}
    \label{fig:massas_dagua}
\end{figure}

%\section{Divisão da PCSE}
% §.4 explicar as frentes e inserir o termo FTP e  FHS
%\hspace{5mm} Considerando que as diferentes propriedades entre as massas intensificarão os gradientes quase-horizontais de densidade, há então a ocorrência de frentes termohalinas na região \citep{castro2008processos}. Sendo assim, \citep{castro1996correntes} define em seu trabalho a presença de duas frentes na região norte da PCSE: a Frente Térmica Profunda (FTP), associada a intrusão da ACAS, separando a água costeira das águas com origem em regiões mais profundas, e a Frente Halina Superficial (FHS), associada ao encontro entre água tropical e água costeira.
\hspace{5mm} Segundo \cite{castro1996correntes}, há presença de duas frentes na PCSE: a Frente Térmica Profunda (FTP) e a Frente Halina Superficial (FHS). A Frente Térmica Profunda, é caracterizada pela intersecção da termoclina com o fundo, formada na região de encontro entre a ACAS e AC e associada a intrusões pelo fundo das águas transportadas pela CB em direção à costa. Já a Frente Halina Superficial é caracterizada como uma frente de quebra da plataforma, embora não esteja posicionada sobre a quebra da plataforma. Está associada à intrusões das águas da CB e separa, na superfície, a AC da AT.

% §.5 classificacao das regiões e sazonalidade
\hspace{5mm} Embasado em trabalhos anteriores, no conhecimento sobre as propriedades termohalinas da região da Plataforma Continental Norte de São Paulo (PCNSP) e análises de dados hidrográficos, \cite{castro1996correntes} propôs a divisão da plataforma em 3 ambientes (Figura~\ref{fig:pcnsp}), com variação sazonal:

\begin{itemize}
    \item Plataforma Continental Interna: localizada entre a costa e FTP, a PCI varia sazonalmente, possuindo uma largura de 10-30km e um limite externo entre as isóbatas de 20-40m no verão e largura de 40-80km com limite externo entre as isóbatas de 50 e 70m no inverno. Embora possua uma variação sazonal, a PCI tende a ter uma temperatura homogênea tanto na vertical quanto na horizontal;
    \item Plataforma Continental Média: compreendida entre FTP e FHS, é bem definida no verão, com a presença da termoclina sazonal, variando de 10-30km da costa até aproximadamente 60-80km. No inverno, a PCM reduz, compreendo uma região de 40-60km a 60-80km, a partir da costa e
    \item Plataforma Continental Externa: prolongando-se da FHS até a quebra da plataforma continental, entre as isóbatas de 70 e 90m. Possui estratificação vertical bem definida, mas com termoclina mais difusa no verão, e possui pequena variação sazonal de suas propriedades.
\end{itemize}

\begin{figure}[!h]
    \centering
    \includegraphics[width=0.65\linewidth]{figuras/divisao_PCNSP.png}
\caption[Divisão da Plataforma Continental de São Paulo]{Esquema da divisão da Plataforma Continental Norte de São Paulo para (a) verão e (b) inverno. IS, MS e OS representam, respectivamente, PCI, PCM e PCE, enquanto que BTB e SHB representam as frentes FTP e FHS. A distância da costa está apresentada em quilômetros e a profundidade em metros. Adaptado de \citep{castro2014summer}.}
    \label{fig:pcnsp}
\end{figure}

% §.6 sazonalidade dos compartimentos
\hspace{5mm} \cite{castro2014summer} idenfitica em seu trabalho que o principal processo da estratiticação da PCI e PCM ao largo de Ubatuba é a variação vertical de densidade, sendo que na PCI é devido a descarga fluvial com águas de baixa salinidade e, na PCM, pela intrusão da ACAS no fundo. Tal variação sazonal pode ser atribuída a outras regiões da PCSE, desde o norte, próximo a Cabo Frio, quanto a sul, próximo a Santa Catarina \citep{cerda2014,brandini2014deep,nogueira2013distribution}.

% §.7 expansão da classificação da plataforma para a PCSE
\hspace{5mm} Desta forma, podemos utilizar a subdivisão da Plataforma Continental em diversas regiões da PCSE \citep{de2003intrusoes,cerda2014}, atentando-se para possíveis particularidades, como no caso de Santa Catarina, onde no verão a subdivisão é adequada, mas no inverno deve-se considerar possíveis intrusões de águas de baixa temperaturas vindas do Rio Grande do Sul \citep{carvalho2010estrutura}.

\section{Circulação na PCSE}
\label{sec:circulation}

% §.8 falar das 4 forçantes: vento, densidade, maré e CB
\hspace{5mm} As correntes na PCSE são forçadas pelos seguintes processos:

\begin{itemize}
    \item Corrente do Brasil (CB): influencia, principalmente, a PCE, ao gerar movimentos devido os meandros e vórtices liberados. Em regiões de plataforma mais estreita, a influência da CB pode ocupar toda a extensão da plataforma \citep{castro2008processos};

    \item Marés: geram correntes perpendiculares à costa, nas regiões mais largas da PCSE, e correntes paralelas à costa em regiões mais estreitas da PCSE \citep{pereira2007numerical}, sendo que a componente M2 é a de energia mais significativa;

    \item Gradientes de Densidade: causado quando águas de origem continental e, portanto, de baixa salinidade, se encontram com as águas da plataforma, de maior salinidade, ocasionando correntes geostróficas paralelas à linha de costa, deixando-a a esquerda do movimento \citep{brink2005coastal} e

    \item Tensão de Cisalhamento do Vento: é a maior forçante de mecanismos de baixa frequência na circulação costeira. O regime de ventos na região Sudeste brasileira é determinado por dois sistemas meterológicos, a Alta Subtropical do Atlântico Sul (ASAS) e Sistemas Frontais \citep{castro1998physical}. A ASAS está relacionada com o regime de vento predominante da PCSE, com ventos de leste e nordeste, que são intensificados no verão. Já os sistemas frontais, ou frentes frias, ocorrem em um período de 6 a 11 dias, formadas ao sul do Brasil e associadas a ventos do quadrante sul \citep{stech1990estudo,stech1992response}.
\end{itemize}

%\begin{figure}[!h]
%    \centering
%    \includegraphics[width=0.8\linewidth]{figuras/campoVento.png}
%\caption[Campo de tensão de cisalhamento do vento.]{Tensão de cisalhamento do vento na região oeste do oceano Atlântico Sul para janeiro e julho, %com a plataforma continental sudesteno centro. Figura extraída de \cite{mazzini2009correntes}.}
%\label{fig:campoVento}
%\end{figure}

% §.9 explicar as forçantes em cada subdivisão
\hspace{5mm} Os movimentos das águas da PCSE são geradas por uma combinação diferente das forçantes citadas, em diferentes regiões da plataforma e em distintas escalas espaciais e temporais \citep{castro1996correntes}. Na plataforma externa, as correntes são forçadas, principalmente, pela Corrente do Brasil \citep{castro2008processos}, com pequenas contribuições da tensão do cisalhamento dos ventos na direção quase paralela às isóbatas locais \citep{dottori2009response}. Já na plataforma média, a forçante predominante é a tensão do cisalhamento dos ventos e a maré, sendo que a influência da Corrente do Brasil só é observada nas regiões mais estretas da plataforma. Por fim, a plataforma interna possui influência das marés, gradientes de densidade e a tensão de cisalhamento do vento.

% §.10 explicar pq focarei somente ne PCI e PCM e somente no vento
\hspace{5mm} Dado que o tema da dissertação do autor será avaliar a influência de ventos anômalos, em eventos raros, sobre a PCSE, a seguir será descrito de forma mais completa a circulação na Plataforma Continental Interna e Média, onde o vento atua com maior influência na circulação.

\section{Circulação na PCI e PCM}

% §.11 principais forcantes na PCI e PCM
\hspace{5mm} Segundo \cite{castro1996correntes}, a circulação em grande parte da plataforma interna é forçada, em diferentes escalas de tempo, pelos ventos, marés e por gradientes de densidade. Sendo que, a maior variância das correntes na PCI e na PCM, foram observadas em duas bandas de frequência, 3-7 e 9-15 dias, associados às oscilações do vento e do nível do mar \citep{castro1998physical}.

% §.12 informacoes sobre PCI (dados in situ e etc)
\hspace{5mm} \cite{castro1996correntes}, ao analisar dados \textit{in situ} na plataforma interna, ao largo de Ubatuba, comprovou a predominância de correntes para SW durante o inverno, com valores típicos da ordem de 0.2 $m.s^{-1}$, havendo eventos de inversão do fluxo para NE, com intensidades da ordem de 0.10 $m.s^{-1}$. Segundo o mesmo autor e demonstrado posteriormente por \cite{dottori2009response}, as componentes paralelas à costa da corrente são essencialmente barotrópica, representando 95$\%$ da variância do primeiro modo ortogonal empírico. Além disso, observou-se que as correntes quase paralelas à costa foram as mais intensas, apresentando uma variabilidade temporal mais energética que as correntes quase perpendiculares.

% jato costeiro, tipico de pci
% \hspace{5mm} jatos costeiros e variabilidade subinercial destes

\hspace{5mm} \cite{valente1999circulaccao} analisou dados correntográficos na plataforma interna, ao largo de Praia Grande e Ubatuba, no litoral Norte de São Paulo. Os pontos a sul da Ilha de São Sebastião apresentaram correntes mais frequentes para N-E, enquanto que os pontos a norte da mesma ilha apresentaram correntes opostas, para S-W. Tal fato foi confirmado por \cite{mazzini2009correntes}, ao analisar correntográficos entre Peruíbe e São Sebastião (SP). Desta forma, pode-se constatar que, durante o verão, há um balanço entre a corrente gerada pela tensão de cisalhamento do vento, para SW, e a corrente gerada pelo gradiente de densidade, para NE, devido a geostrofia.

%-------------------------------------------------------------------------------------------------------

% § transicao entre PCI e PCM, falar sobre castro1996
\hspace{5mm} A principal forçante da circulação da plataforma média são os ventos, havendo coerência significativa entre plataforma interna e média, em oscilações de períodos médios e curtos. A direção predominante dos fluxos na PCM são para SW, com intensidade entre 0.30-0.20 $m.s^{-1}$, havendo inversões para NE mais frequentes do que na PCI. Entretanto, a componente paralela é essencialmente barotrópica e  a componente normal possui um grande cisalhamento vertical, tornando os modos baroclínicos de grande importância \citep{castro1996correntes}.

% §.13 estudos numéricos
\hspace{5mm} Diversos estudos numéricos para avaliar a resposta da plataforma interna aos regimes de ventos foram realizados. \cite{castro1985subtidal} estudou a resposta hidrodinâmica barotrópica da PCSE ao vento climatológico típico de inverno, obtendo como conclusão que as correntes ficam confinadas na plataforma continental com escala de decaimento neperiano de 70-120 km da costa. Ainda neste estudo, foi possível observar que a componente normal à costa é dominada pelo balanço geostrófico e a componente paralela à costa está em balanço friccional, entre a tensão de cisalhamento do vento e a tensão de cisalhamento com o fundo, nas áreas mais costeiras.

\hspace{5mm} As correntes paralelas à costa podem sofrer inversão de sentido, em escala subinercial, devido à passagem de sistemas frontais \citep{stech1992response,dottori2009response}. \cite{coelho2008resposta} estudou a resposta da PCSE a ventos sazonais e sinóticos de verão, obtendo como resultados fluxos para NE na plataforma interna e para SW na plataforma média e externa. O autor utilizou como forçante a passagem de sistemas frontais típicos de verão na região, onde a porção ao sul da Ilha de São Sebastião apresentou a resposta com maiores intensidades a este evento. Com a inversão do regime de ventos na PCSE durante a passagem da frente, observou-se que as correntes na plataforma interna são intensificadas, enquanto que na plataforma média e externa há um enfraquecimento da corrente nas camadas mais superciciais.


% §.17 trabalho do Pedro
\hspace{5mm} \cite{morais2016hidrodinamica}, ao analisar um extenso conjunto de dados e utilizando saídas de modelos numéricos, corrobora com os trabalhos acima apresentados. No entanto, a descarga fluvial ainda não foi devidamente analisada, devido a ausência de dados de vazão dos principais rios e estuários da PCSE, para averiguar seu papel na circulação resultante na PCI e PCM. Ainda assim, espera-se que, em períodos de maior chuva, a descarga fluvial aumente, reduzindo a salinidade das águas da PCI, intensificando o gradiente horizontal de densidade, tendendo a gerar correntes geostróficas que deixam a costa à esquerda do movimento \citep{de2003intrusoes}.


% §.18 componentes de maré de maior importância
\hspace{5mm} \cite{mesquita1983tides} analisaram dados horários de nível do mar para um período de um ano, em Cananéia e Ubatuba, concluindo que as constituintes diurnas dominantes nas estações fora, respectivamente, $O_1$, com 0.110 e 0.109m, e $K_1$, com 0.065 e 0.059. Entre as semi-diurnas, os valores obtidos foram: $M_2$, 0.366 e 0.297m, e $S_2$, com 0.0237 e 0.171m. Todas as constituintes apresentaram rotação anticiclônica, com exceção da $O_1$ e toda a energia proveniente dessas componentes é transportada ao longo da plataforma por ondas de Poincaré e de Kelin \citep{munk1970tides}.

\hspace{5mm} As correntes de maré na PCI são, em geral, uma ordem de grandeza menos intensa do que as correntes geradas pela combinação das forçantes gradiente de densidade e tensão de cisalhamento do vento \citep{castro1996correntes}, sendo que nas proximidades de Santos e no Canal de São Sebastião, \cite{valente1999circulaccao} obteve magnitudes em torno de 0.02 a 0.06 $m.s{-1}$.

\section{Episódios Anômalos}
\label{sec:anomamlousEpisodes}

\hspace{5mm} 

\hspace{5mm} Embora haja períodos em que os ventos predominam de Sul na PCSE, o tempo de
influência deste regime não passa de alguns dias, durante a passagem das frentes frias.
Entretanto, durante o verão de 2014, os ventos associados ao sistema que antes eram de Leste e
Nordeste, passam a ser de Sudoeste, agindo na região costeira por um longo período, como
pode ser observado na Figura~\ref{fig:a701}, onde a direção e intensidade do vento
obtida no \textit{Climate Forecast System Version 2} (CFSv2), durante o período de
01 de Dezembro de 2013 a 28 de Feveiro de 2014. Nota-se que a direção
predominante neste período é do terceiro quadrante (S/SW).

\begin{figure}[!h]
    \centering
    \includegraphics[width=0.65\linewidth]{figuras/windrose_2014CFSv2.png}
\caption[Rosa dos Ventos para Dezembro/2013 a Fevereiro/2014]{Rosa dos ventos para ventos a 10 metros de altura da superfície, extraídos do \textit{Climate Forecast System Version 2} (CFSv2), para o período de Dezembro/2013 a Fevereiro/2014, utilizando-se a conveção meteorológica.}
    \label{fig:a701}
\end{figure}

\hspace{5mm} Com a mudança observada no regime de ventos predominante neste período, a circulação conduzida pelo vento
na plataforma continental interna e média foram alteradas. 
Frente a importância da PCSE, tanto no aspecto de navegação, quanto de
exploração e uso das águas e considerando que as mudanças climáticas poderão
influenciar a frequência de eventos de deslocamento da ASAS, é importante compreender
como a dinâmica da circulação será afetada sob a influência de novos regimes
de vento.

% \hspace{5mm} Com essa mudança no regime de ventos predominante neste
% cenário, a circulação conduzida pelo vento na plataforma muito provavelmente sofreu alterações.
% Frente a importância da PCSE, tanto no aspecto de navegação, quanto de
% exploração e uso das águas e considerando que as mudanças climáticas poderão
% influenciar a frequência de eventos de deslocamento da ASAS, é importante compreender
% como a dinâmica da circulação será afetada sob a influência de novos regimes
% de vento.



\section{Objetivo Geral e Específico}
\label{sec:objectives}

\hspace{5mm} A hipótese científica deste trabalho é que os ventos do quadrante Sul, associados
ao deslocamento da Alta Subtropical do Atlântico Sul para oeste de sua posição
climatológica, alterou de forma efetiva a dinâmica das águas da Plataforma
Continental Sudeste, durante o verão de 2014.

\hspace{5mm} Para tanto, este trabalho tem como objetivo analisar dados de vento de reanálise de banco de dados públicos, a fim de elaborar uma climatologia para os meses de verão (Dezembro, Janeiro e Fevereiro). Esta climatologia será utilizada, então, em um experimento controle para modelar a circulação da Plataforma Continental Interna e Média Sudeste. 
Outros experimentos serão então realizados, utilizando ventos de saída de modelos regionais para o verão de 2014, a fim de comparar a circulação gerada no controle com os cenários de ventos anômalos na região. Mais especificamente, pretende-se:

\begin{itemize}
    \item Determinar a climatologia de ventos para os meses de verão (Dezembro, Janeiro, Fevereiro e Março) na PCSE;
    % \bigskip
    \item Implementar o modelo hidrodinâmico na PCSE para estudar as correntes geradas pelo vento local;
    % \bigskip
    \item Verificar as mudanças geradas pela alteração do regime de ventos.
    % \bigskip
\end{itemize}
