\chapter{Atividades Realizadas}

\hspace{5mm} Nos primeiros 12 meses do curso de Mestrado em Oceanografia com área de concentração em Oceanografia Física, foram cursadas, ao todo, 7 disciplinas (Tabela~\ref{tab:disciplines}), sendo 2 do Núcleo Básico de Oceanografia Física e outras selecionadas com relação ao segmento de estudo deste projeto.

% \begin{tablehere}
% \centering \def\arraystretch{1.5} \small
% \caption{Disciplinas cursadas durante os primeiros 12 meses do curso de Mestrado, com área de Concentração em Oceanografia Física}
% \label{tab:disciplines}
% % \begin{tabular}{|c|c|c|c|}
% \begin{tabular}{|p{2cm}|p{1.5cm}|p{7cm}|p{1.3cm}|}
% \hline
% \textbf{Tipo} & \textbf{Código} & \textbf{Nome}                                                      & \textbf{Créditos} \\ \hline
% Obrigatória   & IOC5815         & Dinâmica de Fluidos Geofísicos I                                   & 6                 \\ \hline
% Obrigatória   & IOC5811         & Dinâmica de Fluidos Geofísicos II                                  & 6                 \\ \hline
% Optativa      & IOC5817         & Métodos de Análise de Dados Quase-Sinóticos em Oceanografia Física & 8                 \\ \hline
% Optativa      & -               & Programa de Aperfeiçoamento ao Ensino (PAE) - FEA USP              & -                 \\ \hline
% Optativa      & IOC5808         & Cinemática e Dinâmica de Estuários                                 & 12                \\ \hline
% Optativa      & IOC5809         & Hidrodinâmica da Plataforma Continental                            & 6                 \\ \hline
% Optativa      & IOC5807         & Modelos Numéricos Aplicados a Processos Costeiros e Oceânicos      & 6                 \\ \hline
% \multicolumn{3}{|r|}{\textbf{Total de Créditos Obtidos:}}                                                     & 44                \\ \hline
% \end{tabular}
% \end{tablehere}

\begin{tablehere}
\centering \def\arraystretch{1.2}
\caption[Disciplinas cursadas no primeiro ano de mestrado]{Disciplinas cursadas durante os primeiros 12 meses do curso de Mestrado, com área de Concentração em Oceanografia Física}
\vspace{0.in}
\label{tab:disciplines}
\begin{tabular}{|c|c|c|c|}
\hline
\textbf{Tipo} & \textbf{Código} & \textbf{Nome}                                                                                                 & \textbf{Créditos} \\ \hline
Obrigatória   & IOC5815         & Dinâmica de Fluidos Geofísicos I                                                                              & 6                 \\ \hline
Obrigatória   & IOC5811         & Dinâmica de Fluidos Geofísicos II                                                                             & 6                 \\ \hline
Optativa      & IOC5817         & \begin{tabular}[c]{@{}c@{}}Métodos de Análise de Dados Quase-Sinóticos \\ em Oceanografia Física\end{tabular} & 8                 \\ \hline
Optativa      & -               & \begin{tabular}[c]{@{}c@{}}Programa de Aperfeiçoamento ao \\ Ensino (PAE) - FEA USP\end{tabular}              & -                 \\ \hline
Optativa      & IOC5808         & Cinemática e Dinâmica de Estuários                                                                            & 12                \\ \hline
Optativa      & IOC5809         & Hidrodinâmica da Plataforma Continental                                                                       & 6                 \\ \hline
Optativa      & IOC5807         & \begin{tabular}[c]{@{}c@{}}Modelos Numéricos Aplicados a Processos \\ Costeiros e Oceânicos\end{tabular}      & 6                 \\ \hline
\multicolumn{3}{|r|}{\textbf{Total de Créditos Obtidos:}}                                                                                                & 44                \\ \hline
\end{tabular}
\end{tablehere}

\bigskip

\hspace{5mm} Assim foram cumpridos 44 créditos no Programa de Pós-Graduação em Oceanografia, 12 créditos a mais do que o exigido pelo propgrama para depósito da dissertação de mestrado. Além das disciplinas, ainda foram realizadas as seguintes atividades no período:

\vspace{-\topsep}
\begin{enumerate}[noitemsep]
    \item Apresentação de Trabalho no XII Simpósio Sobre Ondas, Marés, Engenharia Oceânica e Oceanografia por Satélite (XII OMARSAT);
    \item Revisão bibliográfica da área de estudo, que propiciou uma primeira versão das seções de Introdução e Metodologia da dissertação (Apêndice A e B, respectivamente);
    \item Redação de artigo referente ao Trabalho de Graduação (2016) a ser submetido no 1º semestre de 2018;
    \item Obtenção de dados preliminares e
    \item Submissão de resumo para apresentação de pôster na Ocean Sciences Meeting 2018.
\end{enumerate}

\hspace{5mm} As atividades realizadas seguem o cronograma inicialmente proposto, sem atrasos de qualquer natureza. Para o seguindo ano, segue o cronograma de atividades na próxima seção.